\chapter{Inline ASM}
\def\cst#1{{\color{red}\lstinline|#1|}}
\def\clb#1{{\color{blue}\lstinline|#1|}}
\def\inlineasm{{\color{black}\bf Inline ASM}}

\gcc 可以在\clang 中使用\inlineasm 来达到较底层的操作, 在{\bf Linux Kernel}中使用许多的\inlineasm.\par
\inlineasm 可以分为{\bf Basic Form}和{\bf Extend Form}, 前者看上去比较像\gas 中所使用的汇编语法, %
后者则是在前者的基础上增加了一些功能。

\begin{codelst}[H]
\label{lst:formofasm}
\index{Inline ASM Format}
\caption{Inline ASM 的基本格式}
\begin{clangcode}
__asm__[volatile][goto](ASM Template
                        [: OutputOperands]
                        [: InputOperands ]
                        [: Clobbers      ]
                        [: GotoLabels    ]);
\end{clangcode}
\end{codelst}

%{{{ section: Basic Form
\section{Basic Form}

在\clang 中使用\inlineasm 需要用到\keyword{\_\_asm\_\_}或者\keyword{asm}关键字, %
其中\keyword{asm}是\gcc 的拓展(\gcc 和\llvmclang 都支持这两个关键字), %
推荐优先使用\keyword{\_\_asm\_\_}.\par

\medskip

\begin{codelst}[H]
\caption{A simple inline asm with Basic Form}
\label{lst:firstinlineasm}
\begin{clangcode}
#include<stdio.h>

int main()
{
__asm__ ("push %%rbp\t\n"
         "mov %%rsp, %%rbp");
return 0;
}
\end{clangcode}
\end{codelst}

\reflst{lst:firstinlineasm}是一个简单的{\bf Basic Form}的\inlineasm.%
在{\bf Basic Form}的\inlineasm 中使用 {\bf \lstinline|""|} 包含一个ASM操作, 每个ASM操作使用%
\lstinline|\t\n|进行结束一个ASM操作。\par
{\bf Basic Form} \inlineasm 不能进行\clang 中变量的操作, 存在巨大的局限, %
所以需要{\bf Extend Form} \inlineasm.

%}}} End section: Basic Form

%{{{ section: Extend Form
\section{Extend Form}

\subsection{Qualifier}
\keyword{volatile}和\keyword{goto}可以用于\inlineasm 的修饰.
\keyword{volatile}用于关闭\gcc 对于\inlineasm 的优化, \keyword{goto}(\llvmclang 不支持%
此关键字)用于在\inlineasm 中跳转%
到{\bf GotoLabel}。

\begin{codelst}[H]
\caption{A example of {\bf goto qulifier}}
\label{lst:empofgoto}
\begin{clangcode}
#include<stdio.h>

int main()
{
    int tt_count = 0;
    tt_label:
    if(tt_count <= 4){
        tt_count++;
        printf("This is %d time loop.\n", tt_count);
        __asm__ goto("jmp %l[tt_label]: : : : tt_label");
    }
    return 0;
}
\end{clangcode}
\end{codelst}

\subsection{Operands}

以上部分其实我觉得并不重要, \inlineasm 主要是要理解{\em Operands}, {\em Constraints}和{\em Clobbers}.%
还有不同编译器(\gcc, \llvmclang)与不同得编译参数所得到结果也会有区别. SO ...\par
\bigskip

{\bf Operands}包括{\bf Output  Operands}和{\bf Input Operands}, %
在\inlineasm 中的位置见\reflst{lst:formofasm}. {\bf Operands}的格式为:\newline
\indent [ {\bf [asmSymbolicName]} ] {\bf Constraints} ({\bf cvariableName})\par

{\bf [asmSymbolicName]}是可选的, 编译器默认会依次按照{\bf Operands}的顺序给分配{\bf 0-9}给%
这些{\bf Operands}, 即使{\bf Operands}指定了{\bf [asmSymbolicName]}各个{\bf Operands}还是依照%
{\bf Operands}所在的位置, 只不过指定了{\bf [asmSymbolicName]}的多了一个{\bf refence}。

\begin{codelst}[H]
\caption{The SymbbolicName of Operands}
\label{lst:symname}
\begin{clangcode}
#include<stdio.h>

int main()
{
    int a = 5, b = 10;
    __asm__("addl %[a], %1" // "addl %0, %1" is same
            :
            : [a] "r" (a), "m" (b));
    printf("b is %d.\n", b);
    return 0;
}
\end{clangcode}
\end{codelst}

\reflst{lst:symname}中变量\var{a}, \var{b}依次声明为{\bf Input Operands}. 如果没有给\var{a}指定%
{\bf asmSymbolicName}, 那么{\bf \%0}$\rightarrow$\var{a} {\bf \%1}$\rightarrow$\var{b}. 不过在例子中由于%
\var{a}指定了{\bf [a]}, 所以在ASM部分可以使用{\bf \%[a]} reference to variable \var{a}.

\subsection{Constraints}

\subsection{Clobbers}
%}}} End section: Extend Form
