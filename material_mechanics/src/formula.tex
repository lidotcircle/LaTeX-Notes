\chapter{Formula}

\def\desc{\quad\it}

%{{{ 剪切
\section{剪切}

\subsection{剪切应变能}

{\imp 对微元体做功}:
\begin{equation}
\label{eq:jqdw}
dW = \int_{0}^{\gamma_1}\,\tau\cdot dxdy\cdot d\gamma dz
\end{equation}

\medskip

{\imp 微元体的应变能}:
\begin{equation}
\label{eq:jqdV}
dV_{\epsilon}=dW={\left(\int_{0}^{\gamma_1}\,\tau\cdot d\gamma\right)%
\cdot dV}
\end{equation}

{\imp 应变能密度}, 由公式(\ref{eq:jqdV}):
\begin{equation}
\label{eq:jqv}
v_\epsilon={dV_\epsilon\over dV}={\int_0^{\gamma_1}\,\tau\cdot d\gamma}
\end{equation}

\fbox{\hbox{\vbox{
\begin{quote}
\begin{description}
\item{$dW$:}{\desc 对微元体做功}
\item{$\gamma_1$:}{\desc 剪切最后的剪应变}
\item{$\tau$:}{\desc 剪应力}
\item{$dx\times dy$:}{\desc 剪切面}
\item{$dy$:}{\desc 垂直剪切面}
\item{$dV_\epsilon$:}{\desc 微元体应变能}
\item{$dv_\epsilon$:}{\desc 应变能密度}
\end{description}
\end{quote}
}}}


\subsection{圆轴扭转}

(\ref{eq:suba})由{\imp 圆轴扭转的几何关系和剪切胡克定律}得出, (\ref{eq:subb})由内力关系得出:
\begin{subequations}
\begin{align}
\tau_\rho = G\cdot\rho{d\varphi\over dx}\label{eq:suba}\\
T = \int_{A}\,\rho\tau_\rho\cdot dA\label{eq:subb}
\end{align}
\end{subequations}

将(\ref{eq:suba})代入(\ref{eq:subb})可以得到:\marginnote{\fbox{\vbox{{\tt 式(\ref{eq:ttt})中的$I_p$:}\quad $I_p = \int_A\,\rho^2\,dA$}}}
\begin{equation}
\label{eq:ttt}
T = \int_A\,\rho\cdot G\cdot\rho{d\varphi\over dx}\cdot dA = \cdot{d\varphi\over dx}\int_A\,\rho^2\,dA = G\cdot{d\varphi\over dx}\cdot I_p
\end{equation}

由(\ref{eq:suba})和(\ref{eq:ttt})消去$\displaystyle {d\varphi\over dx}\cdot G$可以得到:
\begin{equation}
\label{eq:jqfff}
T\cdot \left(G\cdot\rho{d\varphi\over dx}\right) = \left(G\cdot{d\varphi\over dx}\cdot I_p\right)\cdot%
\left(\tau_\rho\right)\,\Longrightarrow\,{\tau_\rho = {T\cdot\rho\over I_p}}
\end{equation}

\fbox{\hbox{\vbox{
\begin{quote}
\begin{description}
\item{$T$:}{\desc 外力矩}
\item{$G$:}{\desc 剪切模量}
\item{$\varphi$:}{\desc 截面转角}
\item{$\tau_\rho$:}{\desc 距离圆心为$\rho$处的剪应力}
\end{description}
\end{quote}
}}}
%}}} end 剪切

%{{{ 能量法
\section{能量法}

\subsection{杆件变形能}

{\desc 直杆拉压}:

\begin{subequations}
    \begin{align}
        \text{\imp 杆件变形能:}\quad U = \int_l\,{N_2(x)\,dx\over 2EA}\label{eq:neng1}\\
        \text{\imp 杆件能量密度:}\quad u = {dU\over dV} = {1\over 2}\sigma\varepsilon\label{eq:neng2}
    \end{align}
\end{subequations}

{\desc 纯剪切}:

\begin{equation}
    \label{eq:nengu}
    u = {\tau\over G^2} = {1\over 2}\tau\gamma
\end{equation}

{\desc 扭转}:

\begin{equation}
    \label{eq:nengn}
    U = \int_l\,{T^2(x)\,dx\over2GI_p}
\end{equation}

{\desc 弯曲}:

\begin{equation}
    \label{eq:nengw}
    U = \int_l\,{M^2(x)\,dx\over2EI}
\end{equation}

由式(\ref{eq:neng1})(\ref{eq:nengn})(\ref{eq:nengw})可得线弹性下杆件的广义变形能$U$:
\begin{equation}
    \label{eq:enngg}
    U = {1\over 2}P\cdot\delta
\end{equation}

\subsection{变形能普遍的表达式}

\begin{itemize}
    \item 物体的变形和加载过程无关, 取决与{\imp 最后的受力状态}。
    \item 物体在受力过程始终为{\imp 线弹性}。
\end{itemize}

\indent 物体受$n$个外力(${\vec P_i}, i\in[0, n]$)作用, 各个外力作用处的广义位移分别为${\vec\alpha}_i$($i\in[0, n]$),%
以$\linevec{\beta\cdot{\vec P}}, \beta\in[0, 1]$加载则, 由线弹性可得变形%
为$\linevec{\beta\cdot{\vec\alpha}}, \beta\in[0, 1]$:
\begin{align}
    dW & = \sum_{i=0}^n\,{\left(\beta {\vec P_i}\right)\cdot\left({\vec\alpha}_i d\beta\right)} =%
    \sum_{i=0}^n\,{\left({\vec P_i}{\vec\alpha}_i\beta\right)\,d\beta}\label{eq:gyybnd}\cr
    W & = \int_0^1\,{\sum_{i=0}^n\,{\left({\vec P_i}{\vec\alpha}_i\beta\right)\,d\beta}} = %
    \sum_{i=0}^n\,{1\over 2}{\vec P_i} {\vec\alpha}_i
\end{align}

由线弹性体的力与变形的叠加原理${\vec\alpha_i}$由$n$个外力引起:
\begin{equation}
    \label{eq:ldjz}
    \colvec{\alpha} = \relmat{\delta}\cdot\colvec{P}
\end{equation}

矩阵$\relmat{\delta}$为柔度矩阵。互等定理即为$\delta_{ij} = \delta_{ji}$。
    \index{柔度矩阵}\index{互等定理}

\subsection{卡氏定理}

$\dagger$ 以下部分, ${\vec P_i}$始终不改变方向, 只改变大小。\par

将式(\ref{eq:gyybnd})和(\ref{eq:ldjz})使用求和约定, 并设${\alpha_i} = {\alpha^i}$, %
${P_i} = {P^i}$, ${\delta_{ij}}={\delta^{ij}}={\delta_i^j}={\delta^i_j}$。\par
式(\ref{eq:gyybnd})和(\ref{eq:ldjz})可以重新写成:
\begin{subequations}
    \begin{align}
        W &= U = {1\over2}{\vec P_i}{\vec\alpha^i}\cr
        {\alpha_i} &= \delta_{ij}{P^j}
    \end{align}
\end{subequations}

由上两式可得:
\begin{align}
    {\partial U\over\partial P_k} &= {1\over2}\left({\partial{P_i}\over{\partial P_k}}\cdot{%
    \alpha^i} + {\partial\alpha^i\over{\partial P_k}}\cdot{\vec P_i}\right)\cr
    & = {1\over2}{\alpha^k} + {1\over2}{\delta_m^i\partial{P_m}\over{\partial P_k}}%
    \cdot{P_i}\cr
    & = {1\over2}{\alpha^k} + {1\over2}{\delta_k^i{P_i}}\cr
    & = {\alpha^k}
\end{align}

\subsection{虚功原理}
\label{subsec:xgyl}

\fbox{\vbox{\moveright\parindent\vbox{
\noindent {\imp 虚位移}:

\begin{enumerate}
\item 连续的位移
\item 满足约束条件
\item 满足小位移
\end{enumerate}
}}}

给物体虚位移, 物体所受{\imp 外力做虚功}等于{\imp 造成变形的内力所做虚功}\marginnote{\fbox{\vbox{%
            式(\ref{eq:xgyl})中\newline
            ${\vec P_i}$:集中力\newline
            ${\vec q}$:线荷载\newline
            ${\vec T}$:扭矩\newline
            ${\vec F}$:广义力\newline
            ${S^\star}$:广义变形\newline
            ${\vec N}$:轴力\newline
            ${\vec M}$:内弯矩\newline
            ${\vec Q}$:内剪力\newline
            ${\vec T}$:内扭矩}}}

\begin{align}\label{eq:xgyl}
    {\vec P_i}{v_i^\star} + {\int_l\,{\vec q}v^{\star}\,dx} + {{\vec T^i}{\phi^\star_i}}%
    &= \int_l\,{\vec F}\,d{S^{\star}}\cr
    &= \int\,N\,d(\Delta l)^\star + \int\,M\,d\theta^\star + \int\,Q\,d\gamma^\star + \int\,T\,d\phi^\star
\end{align}

\subsection{莫尔积分}

\subsubsection{单位载荷法}
{\imp 使用单位载荷法求位移}\par
由虚功原理(\ref{subsec:xgyl}), 将要求位置的位移设为$\Delta$, 所求力系的位移设为虚位移,%
再施加单位载荷$1$于所求位置, 可得:
\marginnote{\fbox{\vbox{
            式(\ref{eq:dwzhf})中\newline
            ${\bar N}{\bar M}{\bar Q}{\bar T}$都为单位载荷产生的力
}}}

\begin{equation}
    \label{eq:dwzhf}
    \Delta = {\int\,{\bar N}(x)\,d(\Delta l)} + {\int\,{\bar M}(x)\,d\theta} +%
    {\int\,{\bar Q}(x)\,d\gamma} + {\int\,{\bar T}(x)\,d\phi}
\end{equation}

\subsubsection{莫尔定理}

对于{\imp 线弹性结构}:

\begin{subequations}
    \begin{align}\label{eq:wywf}
        d\theta &= {d\over dx}\left({dv\over dx}\right)dx = {d^2v\over dx^2}dx = {M(x)\over EI}dx\\
        \Delta l &= {Nl\over EA}\\
        d\phi &= {T(x)\over GI_p}dx
    \end{align}
\end{subequations}

由式(\ref{eq:dwzhf})及(\ref{eq:wywf})可以得到:
\begin{equation}\label{eq:medl}
    \Delta = \int_l\,{M(x){\bar M}(x)\,dx\over EI} + \sum_{i=1}^n\,{N_i{\bar N}_il_i\over EA_i}%
    + \int_l\,{T(x){\bar T}(x)\,dx\over GI_p}
\end{equation}

\subsubsection{图乘法}

%}}} end 能量法

%{{{ 荷载,剪力,弯矩
\chapter{内力图}
%}}} end 荷载,剪力,弯矩
