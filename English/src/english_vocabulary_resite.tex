\chapter{Resite}

%{{{ Resite Word
\section{Word}

%{{{ C
\subsection{$\mathcal{C}$}

\vbox{
\label{voc:clause}
{\vocword clause}:\newline\smallskip
\bb
{\vocdef a unit of grammatical organization next below the sentence in rank and in%
traditional grammear said to consist of a subject and predicate.}\newline
\bb[e.g]
{\vocexm "In each sentence above, two clauses are linked by clause-chaining without conjunctions."}\newline
\bb
{\vocdef a particular and separate article, stipulation, or proviso in a treaty, bill, or%
contract.}\newline
\bb[e.g]
{\vocexm "Contracts often have choice-of-law clauses, specifying the law to be applied"}\newline
\bb[synonyms]
{\vocexm section, paragraph, article, subsection, stipulation, condition, proviso, rider}
}

%}}} End C

%{{{ P
\subsection{$\mathcal{P}$}

\vbox{
\label{voc:performative}
{\vocword performative}:\newline\smallskip
\bb[adjective]
{\vocdef relating to or denoting an utterance by means of which the speaker performs%
a paticular act(e.g. I bet, I apologize, I promise)}\newline
\bb[e.g]
{\vocexm "Reather, these are performative utterances, which do not so much say something as do%
something."}\newline
\bb
{\vocdef a performative verb, sentence, or utterance}\newline
\bb[e.g]
{\vocexm "Performative utterances, or performatives, are not true or false and actually perfom%
the action to which they refer."}
}

\vbox{
\label{voc:postposition}
{\vocword postposition}:{\vocchineseword 后置词}\newline\smallskip
\bb
{\vocdef a word or morpheme placed after the word it governs, for example -ward in homeward.}\newline
\bb[e.g]
{\vocexm "It's quite different from English, too, in that it puts the verb at the end%
of the sentence and uses postpositions instead of prepositions."}
}

\vbox{
\label{voc:preposition}
{\vocword preposition}:{\vocchineseword 介词}\newline\smallskip
\bb
{\vocdef a word governing, and usually preceding, a noun or pronoun and expressing a%
relation to another word or element in the clause, as in "the man on the platform,""she arrived%
after dinner,""what did you do it for?"}\newline
\bb[e.g]
{\vocexm "It's quite different from English, too, in that it puts the verb at the end%
of the sentence and uses postpositions instead of prepositions."}
}

%}}} End P

%{{{ S
\subsection{$\mathcal{S}$}

\vbox{
\label{voc:synonym}
{\vocword synonym}:{\vocchineseword 同义词}\newline\smallskip
\bb
{\vocdef a word or phrase that means exactly or nearly the same as another%
word or phrase in the same language, for esample shut is synonym of close.}\newline
\bb[e.g]
{\vocexm "It was common ground that the closest synonym of damage is harm"}
}

%}}} End S

%{{{ U
\subsection{$\mathcal{U}$}

\vbox{
\label{voc:utterance}
{\vocword utterance}:\newline\smallskip
\bb
{\vocdef a spoken word, statement, or vocal sound.}\newline
\bb[e.g]
{\vocexm His bizarre word rhythm and gleeful disregard for punctuation makes even%
his most banai utterances sound dramatic."}\newline
\bb[synonyms]
{\vocexm remark, comment, word, statement, observation, declaration, pronouncement $\ldots$}
}

%}}} End U

%}}}

%{{{ Resite Phrase
\section{Phrase}

%}}}
