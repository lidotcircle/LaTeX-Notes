\chapter{Type}
\index{Type}

变量类型\type{boolean,numeric,pair,path,pen,picture,string,transform}
,还有颜色类型\colortype{rgbcolor,cmykcolor,color}

%{{{ sec:声明
\section{声明}

使用\expression{<type> <var>;}的形式声明变量\expression{<var>}。\par

如\expression{\lstinline$pair m;$},\expression{\lstinline$numeric a;$},%
\expression{\lstinline$string name;$}。

\subsection{newinternal}

使用{\color{blue}\bf newinternal}关键字可以声明新的内置变量,内置变量只可以%
使用{\bf ":="}改变值。
%}}} end sec:声明

%{{{ sec:数组
\section{数组}
\index{Type-array}

\mpost 的数组不声明长度,
声明形如\lstinline$pair pp[];$声明{\bf pp}数组,可以以{\bf pp1,pp2,pp3}%
或者{\bf pp[i]}的形式的调用。
还可以\lstinline$pair p[]p[];$形式的声明,以{\bf p1p1,p1p[i],p.1p.1,p[1]p[1] $\dots$}的形式调用。
%}}} end sec:数组

%{{{ sec:boolean
\section{boolean}
\index{Type-boolean}

\type{boolean}有两个值\constant{true,false},\type{boolean}类型的值由\other{boolean expression}表示。\par
\other{boolean expression}参见\ref{sec:BoolExpression}。
%}}} end sec:boolean

%{{{ sec:numeric
\section{numeric}
\index{Type-numeric}
\label{type:numeric}

%}}} end sec:numeric

%{{{ sec:pair
\section{pair}
\index{Type-pair}
\label{sec:pair}

%}}} end sec:pair

%{{{ sec:path
\section{path}
\index{Type-path}
\label{sec:path}

%}}}

%{{{ sec:pen
\section{pen}
\index{Type-pen}
\label{sec:pen}

%}}} end sec:pen

%{{{ sec:picture
\section{picture}

\subsection{thelabel}
\index{Macro-thelabel}

使用\macro{thelabel}可以返回一个\type{picture}。如\expression{pp = thelabel.bot(%
"Hello world!",origin)}返回\constant{pp},使用\expression{show pp}可以将其打印出来。

\macro{thelabel}的形式:\newline
\indent \expression{thelabel.<suffix>(string | picture,point)}\par

\noindent\lstinline$<suffix>$:\quad{\bf top,bot,lft,rt,ulft,llft,urt,lrt}

与\macro{thelabel}相关的两个宏\macro{label}、\macro{dotlabel}都直接将\type{picture}打印出来。

\subsubsection{thelabel示例}

\begin{mpostcode}[code:thelabel]
outputformat:="eps";
outputtemplate:="%j.eps";
prologues:=3;

beginfig(1);
    string name;
    name := "MyName";
    picture texeq;
    texeq = btex $f(x) = x^2$ etex;

    picture bottest;
    picture toptest;
    picture lfttest;
    lfttest := thelabel.lft(name, origin);
    bottest := thelabel.bot(btex $f(x) = {\sqrt{x}}^3$ etex, origin);
    toptest := thelabel.top(texeq, origin);

    draw lfttest;
    draw bottest;
    draw toptest;
endfig;

end
\end{mpostcode}

\begin{figure}[H]
    \caption{结果图}
    \label{fig:thelabel}
    \centering
    \fbox{\includegraphics{../img/thelabel.eps}}
\end{figure}
%}}} end sec:picture

%{{{ sec:string
\section{string}
\index{Type-string}
\label{sec:string}

%}}} end sec:string

%{{{ sec:transform
\section{transform}
\index{Type-transform}
\label{sec:transform}

%}}}

%{{{ sec:color
\section{color}
\index{Type-color}
\label{sec:color}

\noindent\colortype{color}用于在执行绘图命令时使用\keyword{withcolor}进行改变颜色。\newline
\keyword{withcolor}接受\colortype{rgbcolor}时{\bf map to}\keyword{withrgbcolor}\newline
\keyword{withcolor}接受\colortype{cmykcolor}时{\bf map to}\keyword{withcmykcolor}\newline
\keyword{withcolor}接受\type{numberic}时{\bf map to}\keyword{withgreyscale}\newline
\keyword{withcolor}接受\constant{true}时{\bf map to}\other{current default color}\newline
\keyword{withcolor}接受\constant{false}时{\bf map to}\keyword{withoutcolor}\newline

\subsection{rgbcolor}
\index{Type-rgbcolor}
\label{subsec:rgbcolor}

\begin{mpostcode}[code:rgbcolorex]
    rgbcolor testrgb;
    testrgb:= 0.5red + 0.2green + 0.1blue;
    numeric colorpart[];
    colorpart1:=redpart testrgb;
    colorpart2:=greenpart testrgb;
    colorpart3:=bluepart testrgb;

    for i:=1 upto 3: show colorpart[i]; endfor;
\end{mpostcode}

\keyword{redpart},\keyword{greenpart},\keyword{bluepart}分别用于取得\colortype{rgbcolor}的 (r,g,b)。
\colortype{rgbcolor}的表达式:\newline
\expression{\lstinline$<rgbcolor>c := <numeric>*red + <numeric>*green + <numeric>*blue;$}%
或者\expression{\lstinline$<rgbcolor>c := (<numeric>, <numeric>, <numeric>)$}。
\index{Macro-redpart}\index{Macro-greenpart}\index{Macro-bluepart}

\subsection{cmykcolor}
\index{Type-cmdkcolor}
\label{subsec:cmykcolor}

\other{cmyk = cyan,magenta,yellow,black}\par

与\colortype{rgbcolor}类似的,也有\keyword{cyanpart},\keyword{magentapart},\keyword{yellowpart},%
\keyword{blackpart},表达式不能使用颜色相加的形式。\par
\begin{mpostcode}[code:cmyktest]
cmykcolor testcmyk;
    testcmyk:= (0.1, 0.2, 0.3, 0.4);
    numeric colorpart[];
    colorpart1:=cyanpart testcmyk;
    colorpart2:=magentapart testcmyk;
    colorpart3:=yellowpart testcmyk;
    colorpart4:=blackpart testcmyk;

    for i:=1 upto 4: show colorpart[i]; endfor;
\end{mpostcode}

\subsection{grey}
\label{subsec:grey}

\other{grey}使用\type{numeric}类型表示,范围{\bf [0,1]},从黑到白。
可以使用\macro{greypart}得到\other{grey}颜色的灰度。
\index{Macro-greypart}

%}}} end sec:color
