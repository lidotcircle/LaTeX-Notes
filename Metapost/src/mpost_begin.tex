\chapter{开始使用}

%{{{ sec:基本格式
\section{基本的格式}

\subsection{\mpost 示例}

\begin{mpostcode}[code:exmple]
beginfig(1);
u=1cm;
draw origin -- (u, u);
endfig;

end
\end{mpostcode}

\keyword{\lstinline$beginfig(1)$}用于开始一个新的图形, 其中的“1”称为\constant{charcode}, 图形以\keyword{\lstinline$endfig$}结束。\index{Cons-charcode}%
\constant{origin}表示原点{\bf(0, 0)}。使用{\bf mpost}编译以上代码将生成%
以源文件名(\constant{jobname})为前缀\constant{charcode}后缀的\emph{eps}文件。
\index{Cons-jobname}

\subsection{三个内置变量}

\begin{table}{H}
    \centering
    \begin{tabular}{l | l}
        \hline
        {\bf\large prologues} & 用于决定编译器选择PostScript字体, [0, 3]\cr
        \hline
        {\bf\large outputtemplate} & 输出文件的模板\cr
        \hline
        {\bf\large outputformat} & 输出文件的格式\cr
        \hline
    \end{tabular}
    \caption{三个内置变量}
    \label{tab:outputinternal}
\end{table}

\subsubsection{outputtemplate}

输出文件的模板, 为字符串内置变量。模板包含转义字符{\bf \%\{charcode\}}:\newline
\begin{table}{H}
    \centering
    \begin{tabular}{c | l}
        \hline
        {\bf\large \%j} & jobname\cr
        \hline
        {\bf\large \%c} & beginfig的计数\cr
        \hline
        {\bf\large \%\%} & \%字符\cr
        \hline
        {\bf\large \%\{y | m | d | H | M\}} & 一次表示年, 月, 日, 小时,分钟\cr
        \hline
    \end{tabular}
    \caption{转义字符}
    \label{tab:escapechar}
    \index{转义字符}
\end{table}

\subsubsection{outputformat}

字符串内置变量,输出文件的格式,默认为{\bf eps}, 可选为{\bf eps, svg, png}。

\subsubsection{\mpost 文件输出示例}

\begin{mpostcode}[code:example2]
outputformat:="eps";
outputtemplate:="%j-%c.eps";
prologues:=3;

beginfig(1);
u=1cm;
draw origin -- (u, u);
endfig;

end
\end{mpostcode}

编译文件输出{\bf \{jobname\}-1.eps}。
%}}} end sec:基本格式

%{{{ sec:赋值和推断
\section{赋值和推断}

\subsection{赋值}

\mpost 中的赋值符号是{\bf :=},如给内置变量赋值需要使用\expression{\lstinline$<internalVar>:=<value>$}的形式。{\bf :=}的%
右边只能是确定的值,左边是一个变量。

\subsection{推断}

\mpost 中支持线性表达式值的推断,线性表达式的中使用{\bf =},{\bf =}的左右可以互换。例:

\begin{mpostcode}[code:xianxing]
outputformat:="eps";
outputtemplate:="%j-%c.eps";
prologues:=3;

beginfig(1);
    u = 2cm;
    z1 = (x2 + 2u, y3);
    z2 + z3 = (u, 2u);
    z3 - z2 = (0.5u, u);
    draw z1 -- z2 -- z3 --cycle;
endfig;

end
\end{mpostcode}

无名变量线性推断{\color{red}whatever}
\begin{mpostcode}[code:whatever]
outputformat:="eps";
outputtemplate:="%j-%c.eps";
prologues:=3;

beginfig(1);
    u = 2cm;
    z1 = whatever[origin, (2u, 3u)];
    z1 = whatever[(0, 2u), (2u, 0)];
    draw origin -- z1;
endfig;

end
\end{mpostcode}

将{\color{red} whatever}改为其他的变量一样可以。

\section{单位}

\begin{table}[H]
    \centering
    \begin{tabular}{ c | l}
        \hline
        单位 & Explanation\cr
        \hline
        bp & One PostScript point in bp units\cr
        \hline
        cc & One cicero in bp units [12.79213]\cr
        \hline
        cm & One centimeter in bp units [28.34645]\cr
        \hline
        dd & One didot point in bp units [1.06601]\cr
        \hline
        in & One inch in bp units[72]\cr
        \hline
        mm & One millimeter in bp units [2.83464]\cr
        \hline
        pc & One pica in bp units [11.95517]\cr
        \hline
        pt & One printer’s point in bp units [0.99626]\cr
        \hline
    \end{tabular}
    \index{单位}
    \label{tab:units}
    \caption{单位}
\end{table}

所有的单位都会换算成\unit{bp}进行运算, 如\expression{show 1cm * 1cm}会在编译终端显示\disnumber{803.52127}
%}}} end sec:赋值和推断
