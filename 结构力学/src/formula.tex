\chapter{Formula}

%{{{ 力法
\section{力法}

将超静定结构($n$次超静定)中的多余约束分别由$\displaystyle\linevec{X}$代替,%
再由$\displaystyle\linevec{\Delta} = {\bf 0}$%
得到力法方程($n$元方程)。力法方程的求解即需要先求得除了%
$\displaystyle\linevec{X}$外的方程系数。

\noindent$\dagger$\quad{\bf 约束力$X_i$和位移$\Delta_i$方向都是确定的, 可以由结构%
获得。}

\subsection{力法方程}

\begin{equation}
\label{eq:lffc}
\relmat{\delta}\cdot\colvec{X} + \colvec{\Delta} = \colvec{\bf 0}
\end{equation}

\fbox{\vbox{
式(\ref{eq:lffc})中的$\displaystyle\linevec{X}$为超静定结构%
待求的未知约束力。$\displaystyle\linevec{\Delta}$为待求力对应的位移。
}}

\subsection{解力法方程}

\def\tempee#1{({\bf 0},\ldots, {\bf e}_{#1}, \ldots, {\bf 0})}
设$\linevec{e} = ({\bf 1}, {\bf 1}, \dots, {\bf 1})$, $e_i$的方向和$X_i$方向一致。并设%
${\bf L}_i = \tempee{i}$, 以${\bar N}_i, {\bar M}_i, {\bar T}_i, {\bar Q}_i$分别表示%
单独施加${\bf L}_i$时造成的轴力, 弯矩, 扭矩, 剪力。\par

将结构卸除多余约束转为静定结构后, 单独施加${\bf L}_i$, 再将${\bf L}_j$造成的位移作为%
虚位移施加给结构, 则:\marginnote{\fbox{\vbox{式(\ref{eq:axglf})为${\bf L}_i$所做的虚功,
式(\ref{eq:bxglf})为做虚位移后的内虚功}}}
\begin{subequations}
\begin{align}
W^\star &= \tempee{i}\cdot\colvec{\Delta^\star}\label{eq:axglf}\cr
        &= \tempee{i}\cdot\left(\relmat{\delta}\cdot{\tempee{j}^T}\right)\cr
        &= \delta_{ij}\\
U^\star &= \int_l\,{\bar N}_i(x)\,d{\Delta L}^\star +%
\int_l\,{\bar M}_i(x)\,d{\theta^\star} + \int_l\,{\bar Q}_i(x)\,d{\gamma^\star} +%
\int_l\,{\bar T}_i(x)\,d{\phi^\star}\label{eq:bxglf}\cr
\end{align}
\end{subequations}

结合(\ref{eq:axglf})和(\ref{eq:bxglf}), 并由莫尔积分可得:
\begin{equation}
    \label{eq:deltaij}
    \delta_{ij} = \int_l\,{{\bar N}_i(x)\cdot{\bar N}_j(x)\over EA}\,dx +%
    \int_l\,{{\bar M}_i(x)\cdot{\bar M}_j(x)\over EI}\,dx +%
    \int_l\,{{\bar Q}_i(x)\cdot{\bar Q}_j(x)\over GA}\,dx +%
    \int_l\,{{\bar T}_i(x)\cdot{\bar T}_j(x)\over GI_p}\,dx
\end{equation}

相似的可以得到$\Delta_i$的值:
\begin{equation}
    \label{eq:Deltai}
    \Delta_{i} = \int_l\,{{\bar N}_i(x)\cdot{N}(x)\over EA}\,dx +%
    \int_l\,{{\bar M}_i(x)\cdot{M}(x)\over EI}\,dx +%
    \int_l\,{{\bar Q}_i(x)\cdot{Q}(x)\over GA}\,dx +%
    \int_l\,{{\bar T}_i(x)\cdot{T}(x)\over GI_p}\,dx
\end{equation}

在所求结构{\imp 以弯曲为主}时:
\begin{align}
    \delta_{ij} &= \int_l\,{{\bar M}_i(x)\cdot{\bar M}_j(x)\over EI}\,dx\\
    \Delta_{i} &= \int_l\,{{\bar M}_i(x)\cdot{M}(x)\over EI}\,dx
\end{align}

%}}} end 力法
