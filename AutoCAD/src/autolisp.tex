\chapter{AutoLisp}

%{{{ section: Type
\section{Type}

在\autolisp 中可以使用\lstinline|(type <something>)|来检测对象的类型.

\begin{table}[H]
\caption{Type List\footnote{From autodesk offical site}}
\begin{center}
\begin{tabular}{| >{\bgroup\btype p{0.2\hsize}}<{\egroup} |%
>{\bgroup\bf p{\0.6\hsize}}<{\egroup} |}
\hline
ENAME & Entity names\cr\hline
EXRXSUBR & External ObjectARX applications\cr\hline
FILE & File descriptors\cr\hline
INT & Integers\cr\hline
LIST & Lists\cr\hline
PAGETB & Function paging table\cr\hline
REAL & Floating-point numbers\cr\hline
SAFEARRAY & Safearry\cr\hline
STR & Strings\cr\hline
SUBR & Internal AutoLISP funcitons of funciton loaded from(FAX or VLX) files.%
Function in lISP source files loaded from the AutoCAD Command prompt may also appear as %
SUBR\cr\hline
SYM & Symbols\cr\hline
VARIANT & Variant\cr\hline
USUBR & User-defined functions loaded from Lisp source files.\cr\hline
VLA-object & ActiveX objects\cr\hline
\end{tabular}
\end{center}
\end{table}

\subsection{INT}

\subsection{REAL}

\subsection{STR}

\subsection{LIST}

\subsection{FILE}

\subsection{ENAME}

\subsection{SUBR}

\subsection{VARIANT}
%}}} end section: Type

%{{{ section: Syntax
\section{Syntax And Variable}

\subsection{Condition}

\subsubsection{IF PROGN}

\subsubsection{AND OR NOT}

\subsection{Loop}

\subsubsection{WHILE}

\subsubsection{REPEAT}

\subsection{Function}

\subsubsection{Define A Function}

\subsubsection{Define A Command}

\subsection{Scope of Variable}
%}}} end section: Tyep

%{{{ section: Basic Function
\section{Basic Function}

\subsection{Basic IO}

\subsection{Manipulate STR}

\primf{strcase}, \primf{strcat}, \primf{strlen}, \primf{substr}\par


\subsection{Manipulate LIST}

\subsection{Manipulate Numeric}

\subsection{Others}
%}}} end section: Basic Function

%{{{ section: Communicate with acad
\section{Communicate With ACAD}
%}}} end section: Communicate with acad

%{{{ section: Manipulate autocad object
\section{Manipulate AutoCAD Object}
%}}} end section: Manipulate autocad object
