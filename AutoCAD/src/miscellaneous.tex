\chapter{Miscellaneous}

\newbox\tabcode
\setbox\tabcode\hbox to0.5\hsize{\vbox{\begin{center}\color{red}\tt Hello world!\end{center}}}

%{{{ section: system variable
\section{System variable}

AutoCAD System variable accessed by \primf{getvar} and \primf{setvar}.

\begin{table}[H]
\label{tab:sysvar}
\caption{System variable}
\begin{tabular}{| >{\bgroup\color{sysvar@color}\tt}p{0.15\hsize}<{\egroup} |%
>{\bgroup\btype}p{0.25\hsize}<{\egroup} |%
>{\bgroup\desc@sysvar}p{0.5\hsize}<{\egroup} |}
\hline
{\bf System Variable} & {\bf Type} & {\bf Short Description}\cr
\hline
CLAYER & String & 当前图层(图层名)\cr
\hline
LAYOUTTAB & Interge & 布局面板开关. 0 - 关, 1 - 开\cr
\hline
LIMCHECK & Interge & 能否创建超出\sysvar{LIMMIN}和\sysvar{LIMMAX}的对象. 0 - 能, 1 - 不能.\cr
\hline
LIMMAX & 2D Point & 所有对象的上确界点\cr
\hline
LIMMIN & 2D Point & 所有对象的下确界点\cr
\hline
LWDISPLAY & OFF | ON & 用于控制是否打开线宽显示\cr
\hline
OSMODE & Interge & 控制对象捕捉(Object snap)\cr
\hline
PICKADD & Interge(0 | 1 | 2) & 用于控制是否增加选择的对象\cr
\hline
TOOLTIPS & Interge & 用来开关鼠标浮动在工具上时是否显示工具的帮助信息\cr
\hline
VIEWCTR & 3D Point & 当前窗口的中心点\cr
\hline
VIEWDIR & 3D Point | 2D Point & 窗口视图的角度\cr
\hline
VIEWSIZE & Interge & 窗口的高度\cr
\hline
VSMAX & 2D Point & 当前窗口的上确界点\cr
\hline
VSMIN & 2D Point & 当前窗口的下确界点\cr
\hline
\end{tabular}
\end{table}
%}}} end section: system variable
