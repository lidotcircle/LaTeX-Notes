\chapter{Draft}

this file just for draft some formula or theorem to convinent after writing.

\section{formula}

% 泰勒展开式的证明辅助函数, Taylor展开式余项的证明
\begin{equation}
\phi(t) = R(t) - {R(x)\over{\left(x-x_0\right)^{n+1}}}\left(t - x_0\right)^{n+1}
\end{equation}

% n次插值多项式的误差
\begin{equation}
\phi(t) = R(t) - {R(x)\over{\prod_{i=0}^n\left(x - x_i\right)}}\prod_{i=0}^n\left(t - x_i\right)
\end{equation}

% Newton插值公式
\begin{equation}
\begin{aligned}
N_n(x) = & f(x_0) + f\left[x_0, x_1\right]\left(x - x_0\right) + f\left[x_0, x_1, x_2\right] + \cdots + \cr
& f\left[x_0, x_1, \cdots, x_n\right]\left(x - x_0\right)\left(x - x_1\right)\cdots\left(x - x_{n-1}\right)
\end{aligned}
\end{equation}

% Lagerange插值公式
\begin{subequations}
\begin{align}
L_n\left(x\right) = \sum_{i=0}^n f\left(x_i\right)l_i\left(x\right)\\
l_i\left(x\right) = {\prod\limits_{k\in[0,n], k\ne i}\left(x - x_k\right)%
\over\prod\limits_{k\in[0,n], k\ne i}\left(x_i - x_k\right)}
\end{align}
\end{subequations}

% n次插值公式的余项
\begin{equation}
R_n\left(x\right) = f\left(x\right) - L_n\left(x\right) = %
{f^{(n+1)}(\zeta)\over(n+1)!}\prod_{i=0}^n(x - x_i)
\end{equation}
