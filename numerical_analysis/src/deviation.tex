\chapter{误差分析}

\section{误差的来源}

\fbox{\vtop{
\begin{enumerate}
\item{模型误差}\hfill\break
\indent 数学模型从实际问题经抽象和简化, 并忽略一些次要因素得到的.
\item{参数误差}\hfill\break
\indent 给出的数学模型中往往包含一些根据观测得到的物理量.
\item{截断误差}\hfill\break
\indent 计算中存在只有经过无限过程才能得到的结果, 但实际计算时, 只能用有限的过程来计算.
\item{舍入误差}\hfill\break
\indent 计算中遇到的数据可能位数很多, 也有可能时无穷小数, 但是计算时只能对有限位数进行计算.
\end{enumerate}
}}

%{{{ section: 误差
\section{误差}
% 定义
\newtheorem{absolute_error}{\defn}[section]
\newtheorem{error_limit}[absolute_error]{\defn}
\newtheorem{relative_error}[absolute_error]{\defn}
\newtheorem{relative_error_limit}[absolute_error]{\defn}
\newtheorem{significant_digits}[absolute_error]{\defn}
% 定理
\newtheorem{relative_error_limit_with_numbers_of_significant_digits}{\theo}[section]

\subsection{绝对误差与相对误差}

%绝对误差
\begin{absolute_error}
    \index{绝对误差}\label{definition:absolute_error}
    设$x$是准确值, $\astx$是它的一个近似值, 则称$x-\astx$为$\astx$的{\bf 绝对误差}, 简称{\bf 误差}, 记作$e^\ast$, 即$e^\ast = x - \astx$.
\end{absolute_error}

\medskip

%误差限
\begin{error_limit}
    \index{误差限}
    \label{definition:error_limit}
    设$x$是准确值, $\astx$是它的一个近似值, 称$\astx$的绝对误差的绝对值的上限$\varepsilon^\ast$%
    为$\astx$的{\bf 绝对误差限}, 简称{\bf 误差限}, 即$|e^\ast| = |x-\astx|\le\varepsilon^\ast$.
\end{error_limit}

\bigskip

%相对误差
\begin{relative_error}
    \index{相对误差}
    \label{definition:relative_error}
    设$x$为准确值, $\astx$是它的一个近似值, 称壁纸$e^\ast\over\astx$为近似值$\astx$%
    的{\bf 相对误差}, 记作$e^\ast_r$, 即$e^\ast_r = {e^\ast\over\astx}={x-\astx\over\astx}$.
\end{relative_error}

\medskip

%相对误差限
\begin{relative_error_limit}
    \index{相对误差限}\label{definition:relative_error_limit}
    设$x$为准确值, $\astx$是它的一个近似值, 称$\astx$的相对误差$e^\ast_r$的绝对值上界%
    $\varepsilon^\ast_r$为$\astx$的相对误差限, 即$|e^\ast_r|\le\varepsilon^\ast_r$.
\end{relative_error_limit}

\medskip

设有$\astx$和$y^\ast$是$x$和$y$的近似值, $\varepsilon^\ast(x)$和%
$\varepsilon^\ast(y)$分别为相应的绝对误差限, $x$与$y$的四则运算$x\pm y$, $x\cdot y$和$x/y$的绝对误差限为\hfill\break
\begin{subequations}
    \label{equation:error_limit_arithmetic}
    \index{误差限的四则运算}
    \begin{align}
        \varepsilon^\ast(x\pm y) &= \varepsilon^\ast(x) + \varepsilon^\ast(y)\\
        \varepsilon^\ast(x\cdot y) &= |\astx|\varepsilon^\ast(y) + |y^\ast|\varepsilon^\ast(x)\\
        \varepsilon^\ast(x/y) &= {|\astx|\varepsilon^\ast(y) + |y^\ast|\varepsilon^\ast(x)\over|y^\ast|^2}
    \end{align}
\end{subequations}

对于自变量存在误差时, 函数计算值也有误差. 有$f(x)$的Taylor展开式
\begin{equation*}
    f(x) = \sum_{n=0}^{\infty}{f^{(n)}(\astx)\over n!}(x-\astx)^n
\end{equation*}
可以得到
\begin{equation*}
    \index{函数值的误差限}
    \varepsilon^\ast(y)=\varepsilon^\ast(f(x))\approx|f^\prime(\astx)|\varepsilon^\ast(x)
\end{equation*}

\bigskip

%有效数字
\begin{significant_digits}
    \index{有效数字}
    \label{definition:significant_digits}
    设$x$为准确值, $\astx$是它的一个近似值, 若将$\astx$表示成
    \begin{equation}
        \label{equation:expression_of_decimal_digit}
        \astx = \pm0.\alpha_1\alpha_2\cdots\alpha_n\times10^m
    \end{equation}
    其中$m, n$为整数, $\alpha_1,\alpha_2,\cdots,\alpha_n$为$0~9$之间的数, %
    且$\alpha_1\ne0$, 并且满足关系式
    \begin{equation*}
        |x-\astx|\le{1\over2}\times10^{m-n}
    \end{equation*}
    则称$\astx$具有$n$位{\bf 有效数字}.
\end{significant_digits}

\medskip

%据对误差限和有效数字的关系
\begin{relative_error_limit_with_numbers_of_significant_digits}
    \index{相对误差限与有效数字}
    \label{theorem:relative_error_limit_with_numbers_of_significant_digits}
    设形如\refequa{equation:expression_of_decimal_digit}的近似值$\astx$具有$n$位有效%
    数字, 则其相对误差限为
    \begin{equation}
        |e^\ast_r|\le{1\over2\alpha_1}\times 10^{-(n-1)}
    \end{equation}
    其中$\alpha_1$是$\astx$的第一位有效数字. 反之, 若
    \begin{equation}
        |e^\ast_r|\le{1\over2(\alpha_1+1)}\times 10^{-(n-1)}
    \end{equation}
    则$\astx$至少具有$n$位有效数字.
\end{relative_error_limit_with_numbers_of_significant_digits}

\subsection{条件数与病态问题}
%}}} end section: 误差
