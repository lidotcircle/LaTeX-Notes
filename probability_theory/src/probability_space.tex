\chapter{概率空间}

%{{{ section:随机事件和随机变量
\section{随机事件和随机变量}
% theorem
\newtheorem*{randomizedtrail}{Definition}

\begin{randomizedtrail}[随机事件]
{\bf 随机实验}的结果称为{\bf 随机事件},简称{\bf 事件}
\index{随机事件}
\end{randomizedtrail}
%}}} end section:随机事件和随机变量

%{{{ section:事件的运算
\section{事件的运算}
\theoremstyle{definition}
\newtheorem{monotoneeventcolumn}{Definition}[section]

\indent 以$A$, $B$表示事件。

\begin{enumerate}
\item{\bf 事件的包含:}\quad 事件$A$发生,则事件$B$发生。\equalwith$A\subset B$
\item{\bf 事件的和:}\quad 事件$A$或$B$发生。\equalwith$A\cup B$
\item{\bf 事件的交:}\quad 事件$A$和$B$一起发生。\equalwith$A\cap B$
\item{\bf 事件的差:}\quad 事件$A$发生$B$不发生。\equalwith$A - B$
\item{\bf 事件的逆:}\quad 事件$A$不发生。\equalwith${\bar A} = \Omega/A$
\end{enumerate}

\begin{monotoneeventcolumn}
    有事件列$(A_{n=1}^{\infty})$,$A_i\subset A_j$\equalwith$i<j$,则称$A_{n=1}^{\infty}$为单调递增; 若%
    $A_i\supset A_j$\equalwith$i<j$,则$A_{n=1}^{\infty}$为单调递减。%
    称$\lim\limits_{i\to\infty}A_i$为事件列$A_{n=1}^{\infty}$的极限。%
    两者统称单调事件列
    \index{单调事件列}
\end{monotoneeventcolumn}
%}}} end section:事件的运算

%{{{ section:基本事件空间
\section{基本事件空间}
\newtheorem{basicevent}{Definition}[section]
\newtheorem{basiceventspace}[basicevent]{Definition}

\begin{basicevent}[基本事件]
每次实验必出现一个而且只能出现一个基本事件, 任何两个基本事件不能同时发生。
\index{基本事件}
\end{basicevent}

\begin{basiceventspace}[基本事件空间]
以$\omega$表示基本事件,基本事件空间$\Omega = \{\omega\}$。事件是基本事件空间的子集。只含有穷或者可数个基本事件的基本事件空间称为{\bf 离散的}。
\index{基本事件空间}
\end{basiceventspace}
%}}} end section:基本事件空间

%{{{ section:可测空间
\section{事件$\sigma-$代数$\cdot$可测空间}
\newtheorem{sigmaalgebra}{Definition}[section]
\newtheorem{smallestsigmaalgebraset}[sigmaalgebra]{Definition}

\subsection{事件$\sigma-$代数}

$\Omega$是一个集合,$\mathscr{F}$是$\Omega$的一些子集的集合。

\begin{sigmaalgebra}[$\sigma$代数]
    称$\mathscr{F}$为$\Omega$中的一个$\sigma-$代数:
    \begin{enumerate}
        \item $\Omega\in\mathscr{F}$
        \item $A\in\mathscr{F}$\equalwith${\bar A}\in\mathscr{F}$
        \item 对$(A_{i=1}^{\infty})$($A_i\in\mathscr{F}$), $\bigcup\limits_{i=i}^{\infty}A_i\,\in\mathscr{F}$
    \end{enumerate}
    \index{$\sigma-$代数}
\end{sigmaalgebra}

\noindent$\sigma-$代数的一些性质
\begin{itemize}
    \item$\emptyset\in\mathscr{F}$
    \item 对于$(A_{i=1}^{n})$,有$\bigcup_{i=1}^nA_{i}\in\mathscr{F}$和$\bigcap_{i=1}^{n}A_i\in\mathscr{F}$
    \item 对于$(A_{i=1}^{\infty})$,有$\bigcap_{i=1}^{\infty}A_i\in\mathscr{F}$
\end{itemize}

\subsection{Borel $\sigma-$代数}

\begin{smallestsigmaalgebraset}[给定集系最小$\sigma-$代数]
    \index{给定集系最小$\sigma-$代数}
    给定$\Omega$的集系$\mathscr{H}$的最小$\sigma-$代数$\sigma(\mathscr{H})$:
    \begin{enumerate}
        \item$\mathscr{H}\in\sigma(\mathscr{H})$
        \item$\mathscr{B}$是$\Omega$上的$\sigma-$代数,且$\mathscr{H}\subset\mathscr{B}$。%
            则$\forall\,\mathscr{B}$, 有$\sigma(\mathscr{H})\subset\mathscr{B}$
    \end{enumerate}
\end{smallestsigmaalgebraset}
%}}} end section:可测空间
